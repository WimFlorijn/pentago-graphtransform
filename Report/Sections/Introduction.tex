\section{Introduction}
\label{Introduction}
Graphs can be used to model various problems.
Behaviour or changes to the modelled system can be expressed using graph transformations.
Changes to this model are a series of graph transformations called rules.
These rules are a tree part system, a left side, a right side and a matching graph.
When the two transformations are executed, from left to right, parts of the graph will be removed and new parts will be added.
This way of modelling graph transformations are called push-outs, more about the underlying theorie can be found in %todo recerence.
This report will discuss the modelling of a game called pentago using such graph transformations. 
The modeling is done using a tool called GROOVE.

\vspace{6pt}

Chapter \ref{Pentago} will discuss the game, it's rules and other noteworthy information concerning the playing of the game. After discussing the game we will present our model of the game, mayor rules and noteworthy design choices(Chapter \ref{Design}).