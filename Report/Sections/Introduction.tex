\section{Introduction}
\label{Introduction}
Graphs can be used to model various problems.
Behaviour or changes to the modelled system can be expressed using graph transformations.
The changes to this model are a series of graph transformations called rules.
These rules are a three part system: they consist of a left side, a right side and a matching graph.
When the two transformations are executed, from left to right, parts of the graph may be removed and new parts may be added.
This way of modelling graph transformations are called push-outs.
The underlying mechanisms are researched and discussed by Rensink in \cite{Rensink2006}.
This report discusses the modelling of a board game called pentago using such graphs and graph transformations. 
The modeling is done using a tool called GROOVE \cite{tool-groove}.

\vspace{6pt}

Chapter \ref{Pentago} will discuss the game, it's rules and other noteworthy information concerning the playing of the game. 
After discussing the game we will present our model of the game, mayor rules and noteworthy design choices(Chapter \ref{Design}).
We model different strategies, and review their performance in Chapter \ref{Strategy}. Then we draw a conclusion about the results in Chapter \ref{Conclusion}.
Finally, in chapter \ref{Reflection}, each group member reflects on the course individually.