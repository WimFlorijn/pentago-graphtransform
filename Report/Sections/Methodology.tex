\section{Methodology}
\label{Methodology}
Pentago is a two-player game invented by Tomas Flodén. The game is played on a 6×6 board divided into four 3×3 sub-boards (or quadrants). Taking turns, the two players place a marble of their colo ronto an unoccupied space on the board, and then rotate one of the sub-boards by 90 degrees either clockwise or counter-clockwise. A player wins by getting five of their marbles in a vertical, horizontal or diagonal row. If all 36 spaces on the board are occupied without a row of five being formed then the game is a draw. A multi-player for 3 or 4 players has been developed and features a 9×9 board. This addition is called super-pentago. We plan to model pentago in a scalable fashion so that it can be extended to super-pentago on initialization.

\subsection{Design Choices}
When modelling pentago as a graph we want to take advantage of the matching and transformation abilities of GROOVE. The following functionality needs to be modelled:
\begin{itemize}
	\item A scalable board consisting of sub-boards.
	\item A scalable player count.
	\item A rotation mechanism.
	\item An end-game checking system.
	\item A control programm which describes the game flow.
	\item Different strategies of playing the game.
	\item (Optional) A scalable board generation procedure.
\end{itemize}

On the basis of these requirements and taking into account the functionality which groove has to offer, we have chosen to create the following design for the pentago game.